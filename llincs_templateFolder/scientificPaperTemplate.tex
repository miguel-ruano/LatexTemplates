% This is samplepaper.tex, a sample chapter demonstrating the
% LLNCS macro package for Springer Computer Science proceedings;
% Version 2.20 of 2017/10/04
%
\documentclass[runningheads]{llncs}
%
\usepackage{graphicx}
% Used for displaying a sample figure. If possible, figure files should
% be included in EPS format.
%
% If you use the hyperref package, please uncomment the following line
% to display URLs in blue roman font according to Springer's eBook style:
% \renewcommand\UrlFont{\color{blue}\rmfamily}

\begin{document}
%
\title{Insert Title}
%
%\titlerunning{Abbreviated paper title}
% If the paper title is too long for the running head, you can set
% an abbreviated paper title here
%
\author{Nikolas Bertrand\inst{1}\orcidID{0000-0001-7259-2111}
%
\authorrunning{F. Author et al.}
% First names are abbreviated in the running head.
% If there are more than two authors, 'et al.' is used.
%
\institute{University of Potsdam, 14469 Potsdam ,DE} 
\email{nbertrand@uni-potsdam.de}\\
\url{https://cs.uni-potsdam.de}}
%

\maketitle              % typeset the header of the contribution
%
\begin{abstract}
Abstract is here

\keywords{XX \and YY \and ZZ}
\end{abstract}
%
%
%
\section{Introduction}

Introduction

\section{first Section}

xyz
\begin{enumerate}
	\item this is a list
		
\end{enumerate}

 
\subsection{subsection1}
 hello
 \section{next section} 
bla

%\paragraph{Sample Heading (Fourth Level)}
%The contribution should contain no more than four levels of
%headings. Table~\ref{tab1} gives a summary of all heading levels.



%\noindent Displayed equations are centered and set on a separate
%line.
%\begin{equation}
%x + y = z
%\end{equation}

%\begin{theorem}
%This is a sample theorem. The run-in heading is set in bold, while
%the following text appears in italics. Definitions, lemmas,
%propositions, and corollaries are styled the same way.
%\end{theorem}
%
% the environments 'definition', 'lemma', 'proposition', 'corollary',
% 'remark', and 'example' are defined in the LLNCS documentclass as well.
%
%\begin{proof}
%Proofs, examples, and remarks have the initial word in italics,
%while the following text appears in normal font.
%\end{proof}
%For citations of references, we prefer the use of square brackets
%and consecutive numbers. Citations using labels or the author/year
%convention are also acceptable. The following bibliography provides
%a sample reference list with entries for journal.\cite{1}
%
% ---- Bibliography ----
\bibliographystyle{splncs04}
\bibliography{insert bib file}


\end{document}
